�%=======================================================================
\chapter{\tuprolog{} Libraries}
\label{ch:standard-libraries}
%=======================================================================

Libraries are the means by which \tuprolog{} achieves its
fundamental characteristics of minimality and configurability.
%
The engine is by design choice a minimal, purely-inferential core, which includes only the small set of \emph{built-in}s introduced in the previous Chapter.
%
Any other piece of functionality, in the form of predicates, functors, flags and operators, is delivered by \textit{libraries}, which can be loaded and unloaded to/from the engine at any time: each library can provide a set of predicates, functors and a related theory, which can be used to define new flags and operators.

%Libraries can be loaded at any time in the \tuprolog{} engine, both
%from the Java side, by means of the \texttt{loadLibrary} method of
%the \texttt{Prolog} object representing a \tuprolog{} engine, and
%from the Prolog side, using the \texttt{load\_library/1} predicate.
%%
%For example, suppose you want to exploit some features defined in a
%library whose name is \texttt{ExampleLibrary}. If, on the Java side,
%you want to load the library immediately afterwards building a
%\tuprolog{} engine, you would write the following code, using the
%fully qualified Java class name for the library:
%%
%\begin{verbatim}
%Prolog engine = new Prolog();
%try {
%    engine.loadLibrary("com.example.ExampleLibrary");
%} catch (InvalidLibraryException e) {
%}
%\end{verbatim}
%%

The dynamic loading of libraries can be exploited, for instance, to bound the availability of some functionalities to a specific use context, as in the following example:
%
\begin{verbatim}
% println/1 is defined in ExampleLibrary
run_test(Test, Result) :- run(Test, Result),
                          load_library('ExampleLibrary'),
                          println(Result),
                          unload_library(ExampleLibrary').
\end{verbatim}
%
%The library name can be different from its fully qualified class name.

\noindent The \tuprolog{} distribution include several standard libraries, some of which are loaded by default into any engine--although it is always possible both to create an engine with no pre-loaded libraries, and to create an engine with different (possibly user-defined or third party) pre-loaded libraries.

The fundamental libraries, loaded by default, are the following:
%
\begin{description}

\item[BasicLibrary] (class \texttt{alice.tuprolog.lib.BasicLibrary}) |
    provides the most common Prolog predicates, functors, and operators.
    In order to separate computation and interaction aspects, no I/O
    predicates are included.

\item[ISOLibrary] (class \texttt{alice.tuprolog.lib.ISOLibrary}) |
    provides predicates and functors that are part of the built-in
    section in the ISO standard \cite{iso95}, and are not provided as
    built-ins or by BasicLibrary.

\item[IOLibrary] (class \texttt{alice.tuprolog.lib.IOLibrary}) |
    provides the classic Prolog I/O predicates, except for the ISO-I/O ones.

\item[JavaLibrary] (class \texttt{alice.tuprolog.lib.JavaLibrary}) |
    provides predicates and functors to support multi-paradigm programming
    between Prolog and Java, enabling a complete yet easy access to the
    object-oriented world of Java from \tuprolog{}: features include the
    creation and access of both existing and new objects, classes,
    and resources.
    In the .NET version of \tuprolog{}, this library is replaced\footnote{%
    Actually, integrated: please see Chapter \ref{ch:dotnet} for details.} by \textbf{OOLibrary}, which extends the multi-paradigm programming approach to virtually any language supported by the .NET platform.
    (More on this in Chapter \ref{ch:dotnet}.)
\end{description}

\noindent Other libraries included in the standard \tuprolog{} distribution, but not loaded by default, are the following:

\begin{description}

\item[ISOIOLibrary] (class \texttt{alice.tuprolog.lib.ISOIOLibrary}) |
    extends the above IOLibrary by adding ISO-compliant I/O predicates.

\item[DCGLibrary] (class \texttt{alice.tuprolog.lib.DCGLibrary}) |
    provides support for Definite Clause Grammar, an extension of context
    free grammars used for describing natural and formal languages.

\end{description}

\noindent Further libraries exist that are \textit{not} included in the standard \tuprolog{} distribution, because of their very specific domain: they can be downloaded from the \tuprolog{} site, along with their documentation.
%
Among these, for instance, \texttt{RDFLibrary} (class \texttt{alice.tuprolog.lib.RDFLibrary}) provides predicates and functors to handle RDF documents, etc.

The next Sections present the predicates, functors, operators and flag of each library, as well as the dependencies from other libraries, \textit{except for JavaLibrary}, which is discussed in detail in the context of multi-paradigm programming (Chapters \ref{ch:java-library}, \ref{ch_dotnet}, \ref{ch:java-api}, \ref{ch:howto-develop-libraries} and \ref{ch:p@j}).
%
Throughout this chapter, \texttt{string} means a single-quoted or double-quoted string, as detailed in Chapter \ref{ch:engine}, while \texttt{expr} means an evaluable expression---that is, a term that can be interpreted as a value by some library functors.

%-----------------------------------------------------------------------
\section{BasicLibrary}
\label{sec:basic-library}
%-----------------------------------------------------------------------

%---------------------------------------------------------------------
\subsection{Predicates}
%---------------------------------------------------------------------

%---------------------------------------------------------------------
\subsubsection{Type Testing}
%---------------------------------------------------------------------

\begin{itemize}
 \item \bti{constant/1}\\
    \noindent\bt{constant(X)} is true iff \bt{X} is a constant value.

    \template{constant(@term)}

 \item \bti{number/1}\\
    \noindent\bt{number(X)} is true iff \bt{X} is an integer or a float.

    \template{number(@term)}

 \item \bti{integer/1}\\
    \noindent\bt{integer(X)} is true iff \bt{X} is an integer.

    \template{integer(@term)}

 \item \bti{float/1}\\
    \noindent\bt{float(X)} is true iff \bt{X} is an float.

    \template{float(@term)}

 \item \bti{atom/1}\\
    \noindent\bt{atom(X)} is true iff \bt{X} is an atom.

    \template{atom(@term)}

 \item \bti{compound/1}\\
    \noindent\bt{compound(X)} is true iff \bt{X} is a compound term, that is neither atomic nor a variable.

    \template{compound(@term)}

 \item \bti{var/1}\\
    \noindent\bt{var(X)} is true iff \bt{X} is a variable.

    \template{var(@term)}

 \item \bti{nonvar/1}\\
    \noindent\bt{nonvar(X)} is true iff \bt{X} is not a variable.

    \template{nonvar(@term)}

 \item \bti{atomic/1}\\
    \noindent\bt{atomic(X)} is true iff \bt{X} is atomic (that is is an atom, an integer or a float).

    \template{atomic(@term)}

 \item \bti{ground/1}\\
    \noindent\bt{ground(X)} is true iff \bt{X} is a ground term.

    \template{ground(@term)}

 \item \bti{list/1}\\
    \noindent\bt{list(X)} is true iff \bt{X} is a list.

    \template{list(@term)}
\end{itemize}

%---------------------------------------------------------------------
\subsubsection{Term Creation, Decomposition and Unification}
%---------------------------------------------------------------------

\begin{itemize}

\item \verb|'=..'/2| : \textit{univ}\\
    \noindent\verb|'=..'(Term, List)| is true if \bt{List} is a list consisting of the functor and all arguments of \bt{Term}, in this order.

    \template{'=..'(?term, ?list)}

\item \bti{functor/3}\\
    \noindent\bt{functor(Term, Functor, Arity)} is true if the term \bt{Term} is a compound term, \bt{Functor} is its functor, and \bt{Arity} (an integer) is its arity; or if \bt{Term} is an atom or number equal to \bt{Functor} and \bt{Arity} is 0.

    \template{functor(?term, ?term, ?integer)}

\item \bti{arg/3}\\
    \noindent\bt{arg(N, Term, Arg)} is true if \bt{Arg} is the \bt{N}th
    arguments of \bt{Term} (counting from 1).

    \template{arg(@integer, @compound, -term)}

    \exception{error(instantiation\_error, instantiation\_error(\\
    Goal, ArgNo))} if \texttt{N} or \texttt{Term} are variables. \texttt{Goal} is the goal where the problem occurred, \texttt{ArgNo} indicates the argument that caused the problem (here, \texttt{1} or \texttt{2}).

    \exception{error(type\_error(ValidType, Culprit), type\_error(\\
    Goal, ArgNo, ValidType, Culprit))} if \texttt{N} is not an integer number or \texttt{Term} is not a compound term. \texttt{Goal} is the goal where the problem occurred, \texttt{ArgNo} indicates the argument that caused the problem (here, \texttt{1} or \texttt{2}), \texttt{ValidType} is the expected data type (\texttt{integer} or \texttt{compound}, respectively), \texttt{Culprit} is the wrong term found (either \texttt{N} or \texttt{Term}).

    \exception{error(domain\_error(ValidDomain, Culprit), domain\_\\
    error(Goal, ArgNo, ValidDomain, Culprit))} if \texttt{N} is an int value less than 1. \texttt{Goal} is the goal where the problem occurred, \texttt{ArgNo} indicates the argument that caused the problem (clearly, \texttt{1}), \texttt{ValidDomain} is the expected domain (\texttt{greater\_than\_zero}, respectively), \texttt{Culprit} is the wrong term found (obviously, \texttt{N}).

\item \bti{text\_term/2}\\
    \noindent\bt{text\_term(Text, Term)} is true iff \bt{Text} is the text representation of the term \bt{Term}.

    \template{text\_term(?text, ?term)}

\item \bti{text\_concat/3}\\
    \noindent\bt{text\_concat(Text1, Text2, TextDest)} is
    true iff \bt{TextDest} is the text resulting by appending the text
    \bt{Text2} to \bt{Text1}.

    \template{text\_concat(@string, @string, -string)}

    \exception{error(instantiation\_error, instantiation\_error(\\
    Goal, ArgNo))} if \texttt{Text1} or \texttt{Text2} are variables. \texttt{Goal} is the goal where the problem occurred, \texttt{ArgNo} indicates the argument that caused the problem (here, \texttt{1} or \texttt{2}).

    \exception{error(type\_error(ValidType, Culprit), type\_error(\\
    Goal, ArgNo, ValidType, Culprit))} if \texttt{Text1} or \texttt{Text2} are not atoms. \texttt{Goal} is the goal where the problem occurred, \texttt{ArgNo} indicates the argument that caused the problem (here, \texttt{1} or \texttt{2}),
    \texttt{ValidType} is the expected data type (e.g. \texttt{atom}), \texttt{Culprit} is the wrong term found (either \texttt{Text1} or \texttt{Text2}).

\item \bti{num\_atom/2}\\
    \noindent\bt{num\_atom(Number, Atom)} succeeds iff \bt{Atom} is the atom representation of the number \bt{Number}

    \template{number\_codes(+number, ?atom)}

    \template{number\_codes(?number, +atom)}

    \exception{error(type\_error(ValidType, Culprit), type\_error(\\
    Goal, ArgNo, ValidType, Culprit))} if \texttt{Atom} is a variable and \texttt{Number} is not a number, or, viceversa, if \texttt{Atom} is not an atom. \texttt{Goal} is the goal where the problem occurred, \texttt{ArgNo} indicates the argument that caused the problem (here, \texttt{1} or \texttt{2}), \texttt{ValidType} is the expected data type for the wrong argument (e.g. either \texttt{number} or \texttt{atom}), \texttt{Culprit} is the wrong term found (either \texttt{Number} or \texttt{Atom}).

    \exception{error(domain\_error(ValidType, Culprit), domain\_\\
    error(Goal, ArgNo, ValidDomain, Culprit))} if \texttt{Atom} is an atom that does not represent a number. \texttt{Goal} is the goal where the problem occurred, \texttt{ArgNo} indicates the argument that caused the problem (clearly, \texttt{2}), \texttt{ValidDomain} is the expected domain for the wrong argument (\texttt{num\_atom}), \texttt{Culprit} is the wrong term found (obviously \texttt{Atom}).

\end{itemize}

%---------------------------------------------------------------------
\subsubsection{Occur Check}
%---------------------------------------------------------------------

\noindent When the process of unification takes place between a
variable $S$ and a term $T$, the first thing a Prolog engine should do
before proceeding is to check that $T$ does not contain any occurences
of $S$. This test is known as \emph{occurs check} \cite{ss94} and is
necessary to prevent the unification of terms such as $s(X)$ and $X$,
for which no finite common instance exists. Most Prolog
implementations omit the occurs check from their unification algorithm
for reasons related to speed and efficiency: \tuprolog{} is no
exception. However, they provide a predicate for occurs check
augmented unification, to be used when the programmer wants to never
incur on an error or an undefined result during the process.

\begin{itemize}
\item \bti{unify\_with\_occurs\_check/2}\\
    \noindent\bt{unify\_with\_occurs\_check(X, Y)} is true iff \bt{X} and \bt{Y} are unifiable.
    
    \template{unify\_with\_occurs\_check(?term, ?term)}

\end{itemize}

%---------------------------------------------------------------------
\subsubsection{Expression and Term Comparison}
%---------------------------------------------------------------------
\begin{itemize}
    \item expression comparison (generic template:
    \emph{pred}(@expr, @expr)):\\
        \verb|'=:=', '=\=', '>', '<', '>=', '=<'|;
    
    \item term comparison (generic template:
    \emph{pred}(@term, @term)):\\
         \verb|'==', '\==', '@>', '@<', '@>=', '@=<'|.
\end{itemize}

%---------------------------------------------------------------------
\subsubsection{Finding Solutions}
%---------------------------------------------------------------------
\begin{itemize}
%
\item \bti{findall/3}\\
    \noindent\bt{findall(Template, Goal, List)} is true if and only if
    \bt{List} unifies with the list of values to which a variable X not
    occurring in \bt{Template} or \bt{Goal} would be instantiated
    by successive re-executions of \bt{call(Goal), X = Template} after systematic replacement of all variables in X by new variables.

    \template{\bt{findall(?term, +callable\_term, ?list)}}

    \exception{error(instantiation\_error, instantiation\_error(\\
    Goal, ArgNo))} if \texttt{G} is a variable. \texttt{Goal} is the goal where the problem occurred, \texttt{ArgNo} indicates the argument that caused the problem (obviously, \texttt{1}).

    \exception{error(type\_error(ValidType, Culprit), type\_error(\\
    Goal, ArgNo, ValidType, Culprit))} if \texttt{G} is not a callable goal (for instance, it is a number). \texttt{Goal} is the goal where the problem occurred, \texttt{ArgNo} indicates the argument that caused the problem (here, \texttt{2}), \texttt{ValidType} is the expected data type (\texttt{callable}), \texttt{Culprit} is the wrong term found.

\item \bti{bagof/3}\\
    \noindent\bt{bagof(Template, Goal, Instances)} is true if \bt{Instances} is a non-empty list of all terms such that each unifies with \bt{Template} for a fixed instance W of the variables of \bt{Goal} that are free with respect to \bt{Template}. The ordering of the elements of \bt{Instances} is the order in which the solutions are found.

    \template{bagof(?term, +callable\_term, ?list)}

    \exception{error(instantiation\_error, instantiation\_error(\\
    Goal, ArgNo))} if \texttt{G} is a variable. \texttt{Goal} is the goal where the problem occurred, \texttt{ArgNo} indicates the argument that caused the problem (obviously, \texttt{1}).

    \exception{error(type\_error(ValidType, Culprit), type\_error(\\
    Goal, ArgNo, ValidType, Culprit))} if \texttt{G} is not a callable goal (for instance, it is a number). \texttt{Goal} is the goal where the problem occurred, \texttt{ArgNo} indicates the argument that caused the problem (here, \texttt{2}), \texttt{ValidType} is the expected data type (\texttt{callable}), \texttt{Culprit} is the wrong term found.

\item \bti{setof/3}\\
    \noindent\bt{setof(Template, Goal, List)} is true if \bt{List} is a sorted non-empty list of all terms that each unifies with \bt{Template} for a fixed instance W of the variables of \bt{Goal} that are free with respect to \bt{Template}.

    \template{\bt{setof(?term, +callable\_term, ?list)}}

    \exception{error(instantiation\_error, instantiation\_error(\\
    Goal, ArgNo))} if \texttt{G} is a variable. \texttt{Goal} is the goal where the problem occurred, \texttt{ArgNo} indicates the argument that caused the problem (obviously, \texttt{1}).

    \exception{error(type\_error(ValidType, Culprit), type\_error(\\
    Goal, ArgNo, ValidType, Culprit))} if \texttt{G} is not a callable goal (for instance, it is a number). \texttt{Goal} is the goal where the problem occurred, \texttt{ArgNo} indicates the argument that caused the problem (here, \texttt{2}), \texttt{ValidType} is the expected data type (\texttt{callable}), \texttt{Culprit} is the wrong term found.

\end{itemize}

%---------------------------------------------------------------------
\subsubsection{Control Management}
%---------------------------------------------------------------------
\begin{itemize}
%
\item \bti{(->)/2} : \textit{if-then}\\
\noindent\verb|'->'(If, Then)| is true if and only if \bt{If} is true
and \bt{Then} is true for the first solution of \bt{If}.
%
\item \bti{(;)/2} : \textit{if-then-else}\\
\noindent\verb|';'(Either, Or)| is true iff either \bt{Either} or
\bt{Or} is true.

\item \bti{call/1}\\
    \noindent\bt{call(Goal)} is true if and only if \bt{Goal}
    represents a goal which is true. It is opaque to cut.

    \template{call(+callable\_term)}

    \exception{} the same as the built-in predicate \verb|$call/1|; the exception results to be raised by the auxiliary predicate \texttt{call\_guard(G)}.

\item \bti{once/1}\\
    \noindent\bt{once(Goal)} finds exactly one solution to \bt{Goal}. It is equivalent to \bt{call((Goal, !))} and is opaque to cuts.
    
    \template{once(@goal)}

\item \bti{repeat/0}\\
    Whenever backtracking reaches \noindent\bt{repeat}, execution proceeds forward again through the same clauses as if another alternative has been found.
    
    \template{repeat}

\item \verb|'\+'/1| : \textit{not provable}\\
    \noindent\verb|'\+'(Goal)| is the negation predicate and is opaque to cuts. That is, \verb|'\+'(Goal)| is like \bt{call(Goal)} except that its success or failure is the opposite.
    
    \template{'$\setminus$+'(@goal)}

\item \bti{not/1}\\
    \noindent The predicate \bt{not/1} has the same semantics and implementation as the predicate \verb|'\+'/1|.
    
    \template{not(@goal)}
\end{itemize}

%---------------------------------------------------------------------
\subsubsection{Clause Retrieval, Creation and Destruction}
%---------------------------------------------------------------------

\noindent Every Prolog engine lets programmers modify its logic
database during execution by adding or deleting specific clauses. The
ISO standard \cite{iso95} distinguishes between static and dynamic
predicates: only the latter can be modified by asserting or retracting
clauses. While typically the \emph{dynamic/1} directive is used to
indicate whenever a user-defined predicate is dynamically modifiable,
\tuprolog{} engines work differently, establishing two default
behaviors: library predicates are always of a static kind; every other
user-defined predicate is dynamic and modifiable at runtime.
%
The following list contains library predicates used to manipulate the
knowledge base of a \tuprolog{} engine during execution.

\begin{itemize}

\item \bti{clause/2}\\
    \noindent\bt{clause(Head, Body)} is true iff \bt{Head} matches the head of a dynamic predicate, and \bt{Body} matches its body. The body of a fact is considered to be \bt{true}. \bt{Head} must be at least partly instantiated.

    \template{\bt{clause(@term, -term)}}

    \exception{error(instantiation\_error, instantiation\_error(\\
    Goal, ArgNo))} if \texttt{Head} is a variable. \texttt{Goal} is the goal where the problem occurred, \texttt{ArgNo} indicates the argument that caused the problem (obviously, \texttt{1}).

\item \bti{assert/1}\\
    \noindent\bt{assert(Clause)} is true and adds \bt{Clause} to the end of the database.

    \template{\bt{assert(@term)}}

    \exception{} the same as the built-in predicate \texttt{assertz/1}.

\item \bti{retract/1}\\
    \noindent\bt{retract(Clause)} removes from the knowledge base a dynamic clause that matches \texttt{Clause} (which must be at least partially instantiated). Multiple solutions are given upon backtracking.

    \template{\bt{retract(@term)}}

    \exception{} the same as the built-in predicate \verb|$retract/1|; the exception is raised by the auxiliary predicate \texttt{retract\_guard(Clause)}.

\item \bti{retractall/1}\\
    \noindent\bt{retractall(Clause)} removes from the knowledge base all the dynamic clauses matching with \texttt{Clause} (which must be at least partially instantiated).

    \template{\bt{retractall(@term)}}

    \exception{} the same as the built-in predicate \verb|$retract/1|; the exception is raised by the auxiliary predicate \texttt{retract\_guard(Clause)}.

\end{itemize}

%---------------------------------------------------------------------
\subsubsection{Operator Management}
%---------------------------------------------------------------------
\begin{itemize}

\item \bti{current\_op/3}\\
    \noindent\bt{current\_op(Priority, Type, Name)} is true iff \bt{Priority} is an integer in the range [0, 1200], \bt{Type} is one of the \bt{fx}, \bt{xfy}, \bt{yfx}, \bt{xfx} values and \bt{Name} is an atom, and as side effect it adds a new operator to the engine operator list.

    \template{current\_op(?integer, ?term, ?atom)}
\end{itemize}

%---------------------------------------------------------------------
\subsubsection{Flag Management}
%---------------------------------------------------------------------
\begin{itemize}
\item \bti{current\_prolog\_flag/3}\\
    \noindent\bt{current\_prolog\_flag(Flag,Value)} is true if the value of the flag \bt{Flag} is \bt{Value}
    
    \template{current\_prolog\_flag(?atom,?term)}
\end{itemize}

%---------------------------------------------------------------------
\subsubsection{Actions on Theories and Engines}
%---------------------------------------------------------------------
\begin{itemize}

\item \bti{set\_theory/1}\\
    \noindent\bt{set\_theory(TheoryText)} is true iff \bt{TheoryText}
    is the text representation of a valid \tuprolog{} theory, with the
    side effect of setting it as the new theory of the engine.

    \template{set\_theory(@string)}

    \exception{error(instantiation\_error, instantiation\_error(\\
    Goal, ArgNo))} if \texttt{TheoryText} is a variable. \texttt{Goal} is the goal where the problem occurred, \texttt{ArgNo} indicates the argument that caused the problem (obviously, \texttt{1}).

    \exception{error(type\_error(ValidType, Culprit), type\_error(\\
    Goal, ArgNo, ValidType, Culprit))} if \texttt{TheoryText} is not an atom (i.e. a string). \texttt{Goal} is the goal where the problem occurred, \texttt{ArgNo} indicates the argument that caused the problem (here, \texttt{1}), \texttt{ValidType} is the expected data type (\texttt{atom}), \texttt{Culprit} is the wrong term found.

    \exception{error(syntax\_error(Message), syntax\_error(Goal,\\Line, Position, Message))} if \texttt{TheoryText} is not a valid theory. \texttt{Goal} is the goal where the problem occurred, \texttt{Message} describes the error occurred, \texttt{Line} and \texttt{Position} report the error line and position inside the theory, respectively; if the engine is unable to provide either of them, the corresponding value is set to \texttt{-1}.

\item \bti{add\_theory/1}\\
    \noindent\bt{add\_theory(TheoryText)} is true iff \bt{TheoryText}
    is the text representation of a valid \tuprolog{} theory, with the
    side effect of appending it to the current theory of the engine.

    \template{add\_theory(@string)}

    \exception{error(instantiation\_error, instantiation\_error(\\Goal, ArgNo))} if \texttt{TheoryText} is a variable. \texttt{Goal} is the goal where the problem occurred, \texttt{ArgNo} indicates the argument that caused the problem (obviously, \texttt{1}).

    \exception{error(type\_error(ValidType, Culprit), type\_error(\\
    Goal, ArgNo, ValidType, Culprit))} if \texttt{TheoryText} is not an atom (i.e. a string). \texttt{Goal} is the goal where the problem occurred, \texttt{ArgNo} indicates the argument that caused the problem (here, \texttt{1}), \texttt{ValidType} is the expected data type (\texttt{atom}), \texttt{Culprit} is the wrong term found.

    \exception{error(syntax\_error(Message), syntax\_error(Goal,\\Line, Position, Message))} if \texttt{TheoryText} is not a valid theory. \texttt{Goal} is the goal where the problem occurred, \texttt{Message} describes the error occurred, \texttt{Line} and \texttt{Position} report the error line and position inside the theory, respectively; if the engine is unable to provide either of them, the corresponding value is set to \texttt{-1}.

\item \bti{get\_theory/1}\\
    \noindent\bt{get\_theory(TheoryText)} is true, and
    \bt{TheoryText} is the text representation of the current theory of the engine.

    \template{get\_theory(-string)}

\item \bti{agent/1}\\
    \noindent\bt{agent(TheoryText)} is true, and spawns a
    \tuprolog{} agent with the knowledge base provided as a Prolog
    textual form in \texttt{TheoryText} (the goal is described in the
    knowledge base).

    \template{agent(@string)}

    \exception{error(instantiation\_error, instantiation\_error(\\
    Goal, ArgNo))} if \texttt{TheoryText} is a variable. \texttt{Goal} is the goal where the problem occurred, \texttt{ArgNo} indicates the argument that caused the problem (obviously, \texttt{1}).

    \exception{error(type\_error(ValidType, Culprit), type\_error(\\
    Goal, ArgNo, ValidType, Culprit))} if \texttt{TheoryText} is not an atom (i.e. a string). \texttt{Goal} is the goal where the problem occurred, \texttt{ArgNo} indicates the argument that caused the problem (here, \texttt{1}), \texttt{ValidType} is the expected data type (\texttt{atom}), \texttt{Culprit} is the wrong term found.

\item \bti{agent/2}\\
    \noindent\bt{agent(TheoryText, Goal)} is true, and spawn a
    \tuprolog{} agent with the knowledge base provided as a Prolog
    textual form in \texttt{TheoryText}, and solving the query
    \texttt{Goal} as a goal.

    \template{agent(@string, @term)}

    \exception{error(instantiation\_error, instantiation\_error(\\
    Goal, ArgNo))} if either \texttt{TheoryText} or \texttt{G} is a variable. \texttt{Goal} is the goal where the problem occurred, \texttt{ArgNo} indicates the argument that caused the problem (\texttt{1} or \texttt{2}).

    \exception{error(type\_error(ValidType, Culprit), type\_error(\\
    Goal, ArgNo, ValidType, Culprit))} if \texttt{TheoryText} is not an atom (i.e. a string) or \texttt{G} is not a structure. \texttt{Goal} is the goal where the problem occurred, \texttt{ArgNo} indicates the argument that caused the problem (clearly, \texttt{1} or \texttt{2}), \texttt{ValidType} is the expected data type (\texttt{atom} or \texttt{struct}), \texttt{Culprit} is the wrong term found.

\end{itemize}
%
%---------------------------------------------------------------------
\subsubsection{Spy Events}
%---------------------------------------------------------------------
%
During each demonstration, the engine notifies to interested listeners so-called
{\em spy events}, containing informations on its internal state, such as the
current subgoal being evaluated, the configuration of the execution stack and
the available choice points. The different kinds of spy events currently
corresponds to the different states which the virtual machine realizing the
\tuprolog{}'s inferential core can be found into. \textit{Init} events are
spawned whenever the machine initialize a subgoal for execution; \textit{Call}
events are generated when a choice must be made for the next subgoal to be
executed; \textit{Eval} events represent actual subgoal evaluation; finally,
\textit{Back} events are notified when a backtracking occurs during the
demonstration process.
%
\begin{itemize}
\item \bti{spy/0}\\
    \noindent\bt{spy} is true and enables spy event notification.

    \template{spy}

\item \bti{nospy/0}\\
    \noindent\bt{nospy} is true and disables spy event notification.

    \template{nospy}
\end{itemize}

%---------------------------------------------------------------------
\subsubsection{Auxiliary predicates}
%---------------------------------------------------------------------

\noindent The following predicates are provided by the library's theory.

\begin{itemize}

\item \bti{member/2}\\
    \noindent\bt{member(Element, List)} is true iff \bt{Element} is an element of \bt{List}

    \template{member(?term, +list)}

    \exception{error(type\_error(ValidType, Culprit), type\_error(\\
    Goal, ArgNo, ValidType, Culprit))} if \texttt{List} is not a list. \texttt{Goal} is the goal where the problem occurred, \texttt{ArgNo} indicates the argument that caused the problem (clearly, \texttt{2}), \texttt{ValidType} is the expected data type (\texttt{list}), \texttt{Culprit} is the wrong term found.

\item \bti{length/2}\\
    \noindent\bt{length(List, NumberOfElements)} is true in three different cases: (1) if \bt{List} is instantiated to a list of determinate length, then \bt{Length} will be unified with this length; (2) if \bt{List} is of indeterminate length and \bt{Length} is instantiated to an integer, then \bt{List} will be unified with a list of length \bt{Length} and in such a case the list elements are unique variables; (3) if \bt{Length} is unbound then \bt{Length} will be unified with all possible lengths of \bt{List}.
    
    \template{member(?list, ?integer)}

\item \bti{append/3}\\
    \noindent\bt{append(What, To, Target)} is true iff \bt{Target} list can be obtained by appending the \bt{To} list to the \bt{What} list.

    \template{append(?list, ?list, ?list)}

\item \bti{reverse/2}\\
    \noindent\bt{reverse(List, ReversedList)} is true iff \bti{ReversedList} is the reverse list of \bt{List}.

    \template{reverse(+list, -list)}

    \exception{error(type\_error(ValidType, Culprit), type\_error(\\
    Goal, ArgNo, ValidType, Culprit))} if \texttt{List} is not a list. \texttt{Goal} is the goal where the problem occurred, \texttt{ArgNo} indicates the argument that caused the problem (here, \texttt{1}), \texttt{ValidType} is the expected data type (\texttt{list}), \texttt{Culprit} is the wrong term found.

\item \bti{delete/3}\\
    \noindent\bt{delete(Element, ListSource, ListDest)} is true iff \bt{ListDest} list can be obtained by removing \bt{Element} from the list \bt{ListSource}.

    \template{delete(@term, +list, -list)}

    \exception{error(type\_error(ValidType, Culprit), type\_error(\\
    Goal, ArgNo, ValidType, Culprit))} if \texttt{ListSource} is not a list. \texttt{Goal} is the goal where the problem occurred, \texttt{ArgNo} indicates the argument that caused the problem (here, \texttt{2}), \texttt{ValidType} is the expected data type (\texttt{list}), \texttt{Culprit} is the wrong term found.

\item \bti{element/3}\\
    \noindent\bt{element(Pos, List, Element)} is true iff \bt{Element} is the \bt{Pos}-th element of \bt{List} (element numbering starts from 1).

    \template{element(@integer, +list, -term)}

    \exception{error(type\_error(ValidType, Culprit), type\_error(\\
    Goal, ArgNo, ValidType, Culprit))} if \texttt{List} is not a list. \texttt{Goal} is the goal where the problem occurred, \texttt{ArgNo} indicates the argument that caused the problem (here, \texttt{2}), \texttt{ValidType} is the expected data type (\texttt{list}), \texttt{Culprit} is the wrong term found.

\item \bti{quicksort/3}\\
    \noindent\bt{quicksort(List, ComparisonPredicate, SortedList)} is true iff\\
    \bt{SortedList} contains the same elements as \bt{List}, but sorted according to the criterion defined by \bt{ComparisonPredicate}.

    \template{element(@list, @pred, -list)}
\end{itemize}

%---------------------------------------------------------------------
\subsection{Functors}
%---------------------------------------------------------------------

The following functors for expression evaluation (with the usual semantics) are provided:
\begin{itemize}
    \item unary functors:  \verb|+, -, ~, +|
    \item binary functors:  \verb|+, -, *, \, **, <<, >>, /\, \/|
\end{itemize}

%---------------------------------------------------------------------
\subsection{Operators}
%---------------------------------------------------------------------

The full list of BasicLibrary operators, with their priority and associativity, is reported in Table \ref{tab:basiclibrary-operators}.

Expression comparison operators (\verb|=:=| (equal), \verb|=\=| (different), \verb|>| (greater), \verb|<| (smaller), \verb|>=| (greater or equal), \verb|<=| (smaller or equal)) can raise the following exceptions:

\begin{itemize}

    \item\exception{error(instantiation\_error, instantiation\_error(\\
        Goal, ArgNo))} if any of the arguments is a variable. \texttt{Goal} is the goal where the problem occurred, \texttt{ArgNo} indicates the argument that caused the problem (\texttt{1} or \texttt{2}).

    \item\exception{error(type\_error(ValidType, Culprit), type\_error(\\
        Goal, ArgNo, ValidType, Culprit))} if any of the two arguments is not an evaluable expression. \texttt{Goal} is the goal where the problem occurred, \texttt{ArgNo} indicates the argument that caused the problem (\texttt{1} or \texttt{2}), \texttt{ValidType} is the expected data type (\texttt{evaluable}), \texttt{Culprit} is the wrong term found.

    \item\exception{error(evaluation\_error(Error), evaluation\_error(\\
        Goal, ArgNo, Error))} if an error occurs during the evaluation of any of the two arguments. \texttt{Goal} is the goal where the problem occurred, \texttt{ArgNo} indicates the argument that caused the problem (\texttt{1} or \texttt{2}), \texttt{Error} is the error occurred (e.g. \texttt{zero\_division} in case of a division by zero).

\end{itemize}


\begin{table}[h]
    %
    \begin{center}{\small\tt
    \begin{tabular}{p{2cm}|p{1cm}|p{1cm}}\hline\hline
    Name & Assoc. & Prio. \\ \hline\hline
    :-      &   fx  &   1200 \\
    :-      &   xfx &   1200 \\
    ?-      &   fx  &   1200 \\
    ;       &   xfy &   1100 \\
    ->      &   xfy &   1050 \\
    ,       &   xfy &   1000 \\
    not     &   fy  &   900 \\
    $\setminus$+   &   fy   & 900   \\
    =       &   xfx &   700 \\
    $\setminus$=    &  xfx  &   700 \\
    ==      &   xfx &   700 \\
    $\setminus$==   &  xfx  &   700 \\
    @>      &   xfx & 700   \\
    @<      &   xfx & 700   \\
    @=<    &   xfx & 700   \\
    @>=    &   xfx & 700   \\
    =:=    &   xfx & 700   \\
    =$\setminus$=   &   xfx & 700   \\
    >      &   xfx & 700   \\
    <      &   xfx & 700   \\
    >=      &   xfx & 700   \\
    =<      &   xfx & 700   \\
    is      &   xfx &   700 \\
    =..     &   xfx & 700 \\
    +       &   yfx & 500 \\
    -       &   yfx & 500 \\
    $/\setminus$    &   yfx &   500 \\
    $\setminus/$    &   yfx &   500 \\
    $\ast$  &   yfx & 400 \\
    /       &   yfx & 400 \\
    //      &   yfx & 400 \\
    >>      &   yfx & 400 \\
    <<      &   yfx & 400 \\
    >>      &   yfx & 400 \\
    $\ast$$\ast$  &   xfx & 200 \\
    \textasciicircum  &   xfy & 200 \\
    $\setminus$$\setminus$      &   fx & 200 \\
    -       &   fy & 200 \\
    \hline\hline
    \end{tabular}
    }\end{center}
    \caption{BasicLibrary operators.}\label{tab:basiclibrary-operators}
\end{table}

\clearpage

%-----------------------------------------------------------------------
\section{ISOLibrary}
\label{sec:iso-library}
%-----------------------------------------------------------------------

\noindent \emph{Library Dependencies}: BasicLibrary.

This library contains all the predicates and functors of the Prolog ISO standard and that are not provided directly at the \tuprolog{} core or at the BasicLibrary levels.

%---------------------------------------------------------------------
\subsection{Predicates}
%---------------------------------------------------------------------

%---------------------------------------------------------------------
\subsubsection{Type Testing}
%---------------------------------------------------------------------

\begin{itemize}
\item \bti{bound/1}\\
    \noindent\bt{bound(Term)} is a synonym for the \bt{ground/1} predicate defined in BasicLibrary.

    \template{bound(+term)}

\item \bti{unbound/1}\\
    \noindent\bt{unbound(Term)} is true iff \bt{Term} is not a ground term.

    \template{unbound(+term)}
\end{itemize}

%---------------------------------------------------------------------
\subsubsection{Atoms Processing}
%---------------------------------------------------------------------

\begin{itemize}

\item \bti{atom\_length/2}\\
    \noindent\bt{atom\_length(Atom, Length)} is true iff the integer \bt{Length} equals the number of characters in the name of atom \bt{Atom}.

    \template{atom\_length(+atom, ?integer)}

    \exception{error(instantiation\_error, instantiation\_error(\\
    Goal, ArgNo))} if \texttt{Atom} is a variable. \texttt{Goal} is the goal where the problem occurred, \texttt{ArgNo} indicates the argument that caused the problem (clearly, \texttt{1}).

    \exception{error(type\_error(ValidType, Culprit), type\_error(\\
    Goal, ArgNo, ValidType, Culprit))} if \texttt{Atom} is not an atom. \texttt{Goal} is the goal where the problem occurred, \texttt{ArgNo} indicates the argument that caused the problem (here, \texttt{1}), \texttt{ValidType} is the expected data type (\texttt{atom}), \texttt{Culprit} is the wrong term found.

\item \bti{atom\_concat/3}\\
    \noindent\bt{atom\_concat(Start, End, Whole)} is true iff the \bt{Whole} is the atom obtained by concatenating the characters of \bt{End} to those of \bt{Start}. If \bt{Whole} is instantiated, then all decompositions of \bt{Whole} can be obtained by backtracking.
    
    \template{atom\_concat(?atom, ?atom, +atom)}
    
    \template{atom\_concat(+atom, +atom, -atom)}

\item \bti{sub\_atom/5}\\
    \noindent\bt{sub\_atom(Atom, Before, Length, After, SubAtom)} is true iff \bt{Sub}-\bt{Atom} is the sub atom of \bt{Atom} of length \bt{Length} that appears with \bt{Before} characters preceding it and \bt{After} characters following. It is re-executable.

    \template{sub\_atom(+atom, ?integer, ?integer, ?integer, ?atom)}

    \exception{error(type\_error(ValidType, Culprit), type\_error(\\
    Goal, ArgNo, ValidType, Culprit))} if \texttt{Atom} is not an atom. \texttt{Goal} is the goal where the problem occurred, \texttt{ArgNo} indicates the argument that caused the problem (here, \texttt{1}), \texttt{ValidType} is the expected data type (\texttt{atom}), \texttt{Culprit} is the wrong term found.

\item \bti{atom\_chars/2}\\
    \noindent\bt{atom\_chars(Atom,List)} succeeds iff \bt{List} is a list whose elements are the one character atoms that in order make up \bt{Atom}.

    \template{atom\_chars(+atom, ?character\_list)}
    
    \template{atom\_chars(-atom, ?character\_list)}

    \exception{error(type\_error(ValidType, Culprit), type\_error(\\
    Goal, ArgNo, ValidType, Culprit))} if \texttt{Atom} is a variable and \texttt{List} is not a list, or, conversely, \texttt{List} is a variable and \texttt{Atom} is not an atom. \texttt{Goal} is the goal where the problem occurred, \texttt{ArgNo} indicates the argument that caused the problem (either \texttt{1} or \texttt{2}), \texttt{ValidType} is the expected data type (\texttt{atom} or \texttt{list}, respectively), \texttt{Culprit} is the wrong term found.

\item \bti{atom\_codes/2}\\
    \noindent\bt{atom\_codes(Atom, List)} succeeds iff \bt{List} is a list whose elements are the character codes that in order correspond to the characters that make up \bt{Atom}.
    
    \template{atom\_codes(+atom, ?character\_code\_list)}
    
    \template{atom\_chars(-atom, ?character\_code\_list)}

\item \bti{char\_code/2}\\
    \noindent\bt{char\_code(Char, Code)} succeeds iff \bt{Code} is a the character code that corresponds to the character \bt{Char}.

    \template{char\_code(+character, ?character\_code)}

    \template{char\_code(-character, +character\_code)}

    \exception{error(type\_error(ValidType, Culprit), type\_error(\\
    Goal, ArgNo, ValidType, Culprit))} if \texttt{Code} is a variable and \texttt{Char} is not a character (that is, an atom of length 1), or, conversely, \texttt{Char} is a variable and \texttt{Code} is not an integer. \texttt{Goal} is the goal where the problem occurred, \texttt{ArgNo} indicates the argument that caused the problem (either \texttt{1} or \texttt{2}), \texttt{ValidType} is the expected data type (\texttt{character} or \texttt{integer}, respectively), \texttt{Culprit} is the wrong term found.

\item \bti{number\_chars/2}\\
    \noindent\bt{number\_chars(Number, List)} succeeds iff \bt{List} is a list whose elements are the one character atoms that in order make up \bt{Number}.
    
    \template{number\_chars(+number, ?character\_list)}
    
    \template{number\_chars(-number, ?character\_list)}

\item \bti{number\_codes/2}\\
    \noindent\bt{number\_codes(Number, List)} succeeds iff \bt{List} is a list whose elements are the codes for the one character atoms that in order make up \bt{Number}.
    
    \template{number\_codes(+number,?character\_code\_list)}
    
    \template{number\_codes(-number,?character\_code\_list)}

\end{itemize}

%---------------------------------------------------------------------
\subsection{Functors}
%---------------------------------------------------------------------

\begin{itemize}
    \item Trigonometric functions: \bt{sin(+expr)}, \bt{cos(+expr)}, \bt{atan(+expr)}.
    %
    \item Logarithmic functions: \bt{exp(+expr)}, \bt{log(+expr)}, \bt{sqrt(+expr)}.
    %
    \item Absolute value functions: \bt{abs(+expr)}, \bt{sign(+Expr)}.
    %
    \item Rounding functions: \bt{floor(+expr)},
    \bt{ceiling(+expr)}, \bt{round(+expr)}, \bt{truncate(+expr)},
    \bt{float(+expr)}, \bt{float\_integer\_part(+expr)},\\\bt{float\_fractional\_part(+expr)}.
    %
    \item Integer division functions:
    \bt{div(+expr, +expr)}, \bt{mod(+expr, +expr)}, \bt{rem(+expr, +expr)}.
\end{itemize}

%---------------------------------------------------------------------
\subsection{Operators}
%---------------------------------------------------------------------

The full list of ISOLibrary operators, with their priority and associativity, is reported in Table \ref{tab:isolibrary-operators}.

\begin{table}[h]
    %
    \begin{center}{\small\tt
    \begin{tabular}{p{2cm}|p{1cm}|p{1cm}}\hline\hline
    Name & Assoc. & Prio. \\ \hline
    mod   & yfx & 400\\
    div   & yfx & 300\\
    rem   & yfx & 300\\
    sin   & fx & 200\\
    cos   & fx & 200\\
    sqrt  & fx & 200\\
    atan  & fx & 200\\
    exp   & fx & 200\\
    log   & fx & 200\\
    \hline\hline
    \end{tabular}
    }\end{center}
    \caption{ISOLibrary operators.}\label{tab:isolibrary-operators}
\end{table}

%---------------------------------------------------------------------
\subsection{Flags}
%---------------------------------------------------------------------

The full list of ISOLibrary flags, with their admissible and default values, is reported in Table \ref{tab:isolibrary-flags}.

\begin{table}[h]
    %
    \begin{center}{\small\tt
    \begin{tabular}{p{4.8cm}|p{2.2cm}|p{2.2cm}|p{1.8cm}}\hline\hline
        Flag Name   & Possible Values & Default Value & Modifiable\\ \hline\hline
        bounded          & {true}          &  true        & no\\
        max\_integer     & {2147483647}    &  2147483647  & no\\
        min\_integer     & {-2147483648}   &  -2147483648 & no \\
        integer\_rounding\_function & down & down         & no\\
        char\_conversion & {off}           & off          & no\\
        debug            & {off}           & off          & no\\
        max\_arity       & {2147483647}    & 2147483647   & no\\
        undefined\_predicates & fail       & fail         & no\\
        double\_quotes  &  {atom}          & atom         & no\\
    \hline\hline
    \end{tabular}
    }\end{center}
    \caption{ISOLibrary flags. Any tentative to modify unmodifiable flags will result into a \texttt{permission\_error} exception.}\label{tab:isolibrary-flags}
\end{table}

%-----------------------------------------------------------------------
\section{IOLibrary}
\label{sec:io-library}
%-----------------------------------------------------------------------

\noindent \emph{Library Dependencies}: BasicLibrary.

The IOLibrary defines the classical Prolog I/O predicates; further ISO-compliant I/O predicates are provided by ISOIOLibrary (Section \ref{sec:isoio-library}).

%---------------------------------------------------------------------
\subsection{Predicates}
%---------------------------------------------------------------------

%---------------------------------------------------------------------
\subsubsection{General I/O}
%---------------------------------------------------------------------

\begin{itemize}

\item \bti{see/1}\\
    \noindent\bt{see(StreamName)} is used to create/open an input stream; the predicate is true iff \bt{StreamName} is a string representing the name of a file to be created or accessed as input stream, or the string \texttt{stdin} selecting current standard input as input stream.

    \template{see(@atom)}

    \exception{error(instantiation\_error, instantiation\_error(\\
    Goal, ArgNo))} if \texttt{StreamName} is a variable. \texttt{Goal} is the goal where the problem occurred, \texttt{ArgNo} indicates the argument that caused the problem (obviously, \texttt{1}).

    \exception{error(type\_error(ValidType, Culprit), type\_error(\\
    Goal, ArgNo, ValidType, Culprit))} if \texttt{StreamName} is not an atom. \texttt{Goal} is the goal where the problem occurred, \texttt{ArgNo} indicates the argument that caused the problem (here, \texttt{1}), \texttt{ValidType} is the expected data type (\texttt{atom}), \texttt{Culprit} is the wrong term found.

    \exception{error(domain\_error(ValidDomain, Culprit), domain\_\\
    error(Goal, ArgNo, ValidDomain, Culprit))} if \texttt{StreamName} is not the name of an accessible file. \texttt{Goal} is the goal where the problem occurred, \texttt{ArgNo} indicates the argument that caused the problem (clearly, \texttt{1}), \texttt{ValidDomain} is the expected domain (\texttt{stream}), \texttt{Culprit} is the wrong term found.

\item \bti{seen/0}\\
    \noindent\bt{seen} is used to close the input stream previously opened; the predicate is true iff the closing action is possible

    \template{seen}

\item \bti{seeing/1}\\
    \noindent\bt{seeing(StreamName)} is true iff \texttt{StreamName} is the name of the stream currently used as input stream.

    \template{seeing(?term)}

\item \bti{tell/1}\\
    \noindent\bt{tell(StreamName)} is used to create/open an output stream; the predicate is true iff \bt{StreamName} is a string representing the name of a file to be created or accessed as output stream, or the string \texttt{stdout} selecting current standard output as output stream.

    \template{tell(@atom)}

    \exception{error(instantiation\_error, instantiation\_error(\\
    Goal, ArgNo))} if \texttt{StreamName} is a variable. \texttt{Goal} is the goal where the problem occurred, \texttt{ArgNo} indicates the argument that caused the problem (obviously, \texttt{1}).

    \exception{error(type\_error(ValidType, Culprit), type\_error(\\
    Goal, ArgNo, ValidType, Culprit))} if \texttt{StreamName} is not an atom. \texttt{Goal} is the goal where the problem occurred, \texttt{ArgNo} indicates the argument that caused the problem (here, \texttt{1}), \texttt{ValidType} is the expected data type (\texttt{atom}), \texttt{Culprit} is the wrong term found.

    \exception{error(domain\_error(ValidDomain, Culprit), domain\_\\
    error(Goal, ArgNo, ValidDomain, Culprit))} if \texttt{StreamName} is not the name of an accessible file. \texttt{Goal} is the goal where the problem occurred, \texttt{ArgNo} indicates the argument that caused the problem (clearly, \texttt{1}), \texttt{ValidDomain} is the expected domain (\texttt{stream}), \texttt{Culprit} is the wrong term found.

\item \bti{told/0}\\
    \noindent\bt{told} is used to close the output stream previously opened; the predicate is true iff the closing action is possible.

    \template{told}

\item \bti{telling/1}\\
    \noindent\bt{telling(StreamName)} is true iff \texttt{StreamName} is the name of the stream currently used as input stream.

    \template{telling(?term)}

\item \bti{put/1}\\
    \noindent\bt{put(Char)} puts the character \bt{Char} on current output stream; it is true iff the operation is possible.

    \template{put(@char)}

    \exception{error(instantiation\_error, instantiation\_error(\\
    Goal, ArgNo))} if \texttt{Char} is a variable. \texttt{Goal} is the goal where the problem occurred, \texttt{ArgNo} indicates the argument that caused the problem (obviously, \texttt{1}).

    \exception{error(type\_error(ValidType, Culprit), type\_error(\\
    Goal, ArgNo, ValidType, Culprit))} if \texttt{Char} is not a character, i.e. an atom of length 1. \texttt{Goal} is the goal where the problem occurred, \texttt{ArgNo} indicates the argument that caused the problem (\texttt{1}), \texttt{ValidType} is the expected data type (\texttt{char}), \texttt{Culprit} is the wrong term found.

    \exception{error(permission\_error (Operation, ObjectType,\\
    Culprit), permission\_error(Goal, Operation, ObjectType,\\
    Culprit, Message))} if it was impossible to write on the output stream. \texttt{Goal} is the goal where the problem occurred, \texttt{Operation} is the operation to be performed (here, \texttt{output}), \texttt{ObjectType} is the type of the target object (\texttt{stream}), \texttt{Culprit} is the name of the output stream, and \texttt{Message} provides extra info about the occurred error.

\item \bti{get0/1}\\
    \noindent\bt{get0(Value)} is true iff \bt{Value} is the next character (whose code can span on the entire ASCII codes) available from the input stream, or -1 if no characters are available; as a side effect, the character is removed from the input stream.

    \template{get0(?charOrMinusOne)}

    \exception{error(permission\_error (Operation, ObjectType,\\
    Culprit), permission\_error(Goal, Operation, ObjectType,\\
    Culprit, Message))} if it was impossible to read from the input stream. \texttt{Goal} is the goal where the problem occurred, \texttt{Operation} is the operation to be performed (here, \texttt{input}), \texttt{ObjectType} is the type of the target object (\texttt{stream}), \texttt{Culprit} is the name of the input stream, and \texttt{Message} provides extra info about the occurred error.

\item \bti{get/1}\\
    \noindent\bt{get(Value)} is true iff \bt{Value} is the next character (whose code can span on the range 32..255 as ASCII codes) available from the input stream, or -1 if no characters are available; as a side effect, the character (with all the characters that precede this one not in the range 32..255) is removed from the input stream.

    \template{get(?charOrMinusOne)}

    \exception{error(permission\_error (Operation, ObjectType,\\
    Culprit), permission\_error(Goal, Operation, ObjectType,\\
    Culprit, Message))} if it was impossible to read from the input stream. \texttt{Goal} is the goal where the problem occurred, \texttt{Operation} is the operation to be performed (here, \texttt{input}), \texttt{ObjectType} is the type of the target object (\texttt{stream}), \texttt{Culprit} is the name of the input stream, and \texttt{Message} provides extra info about the occurred error.

\item \bti{tab/1}\\
    \noindent\bt{tab(NumSpaces)} inserts \bt{NumSpaces} space characters (ASCII code 32) on output stream; the predicate is true iff the operation is possible.

    \template{tab(+integer)}

    \exception{error(instantiation\_error, instantiation\_error(\\
    Goal, ArgNo))} if \texttt{NumSpaces} is a variable. \texttt{Goal} is the goal where the problem occurred, \texttt{ArgNo} indicates the argument that caused the problem (obviously, \texttt{1}).

    \exception{error(type\_error(ValidType, Culprit), type\_error(\\
    Goal, ArgNo, ValidType, Culprit))} if \texttt{NumSpaces} is not an integer number. \texttt{Goal} is the goal where the problem occurred, \texttt{ArgNo} indicates the argument that caused the problem (here, \texttt{1}), \texttt{ValidType} is the expected data type (\texttt{integer}), \texttt{Culprit} is the wrong term found.

    \exception{error(permission\_error (Operation, ObjectType,\\
    Culprit), permission\_error(Goal, Operation, ObjectType,\\
    Culprit, Message))} if it was impossible to write on the output stream. \texttt{Goal} is the goal where the problem occurred, \texttt{Operation} is the operation to be performed (here, \texttt{output}), \texttt{ObjectType} is the type of the target object (\texttt{stream}), \texttt{Culprit} is the name of the output stream, and \texttt{Message} provides extra info about the occurred error.

\item \bti{read/1}\\
    \noindent\bt{read(Term)} is true iff \bt{Term} is Prolog term available from the input stream. The term must ends with the \emph{.} character; if no valid terms
    are available, the predicate fails. As a side effect, the term is removed from the input stream.

    \template{read(?term)}

    \exception{error(permission\_error (Operation, ObjectType,\\
    Culprit), permission\_error(Goal, Operation, ObjectType,\\
    Culprit, Message))} if it was impossible to read from the input stream. \texttt{Goal} is the goal where the problem occurred, \texttt{Operation} is the operation to be performed (here, \texttt{input}), \texttt{ObjectType} is the type of the target object (\texttt{stream}), \texttt{Culprit} is the name of the input stream, and \texttt{Message} provides extra info about the occurred error.

    \exception{error(syntax\_error(Message), syntax\_error(Goal,\\
    Line, Position, Message))} if a syntax error occurred when reading from the input stream. \texttt{Goal} is the goal where the problem occurred, \texttt{Message} is the string read from the input that caused the error, while \texttt{Line} and \texttt{Position} are not applicable in this case and therefore default to \texttt{-1}.

\item \bti{write/1}\\
    \noindent\bt{write(Term)} writes the term \bt{Term} on current output stream. The predicate fails if the operation is not possible.

    \template{write(@term)}

    \exception{error(instantiation\_error, instantiation\_error(\\
    Goal, ArgNo))} if \texttt{Term} is a variable. \texttt{Goal} is the goal where the problem occurred, \texttt{ArgNo} indicates the argument that caused the problem (obviously, \texttt{1}).

    \exception{error(permission\_error (Operation, ObjectType,\\
    Culprit), permission\_error(Goal, Operation, ObjectType,\\
    Culprit, Message))} if it was impossible to write on the output stream. \texttt{Goal} is the goal where the problem occurred, \texttt{Operation} is the operation to be performed (here, \texttt{output}), \texttt{ObjectType} is the type of the target object (\texttt{stream}), \texttt{Culprit} is the name of the output stream, and \texttt{Message} provides extra info about the occurred error.

\item \bti{print/1}\\
    \noindent\bt{print(Term)} writes the term \bt{Term} on current output stream, removing apices if the term is an atom representing a string. The predicate fails if the operation is not possible.

    \template{print(@term)}

    \exception{error(instantiation\_error, instantiation\_error(\\
    Goal, ArgNo))} if \texttt{Term} is a variable. \texttt{Goal} is the goal where the problem occurred, \texttt{ArgNo} indicates the argument that caused the problem (obviously, \texttt{1}).

    \exception{error(permission\_error (Operation, ObjectType,\\
    Culprit), permission\_error(Goal, Operation, ObjectType,\\
    Culprit, Message))} if it was impossible to write on the output stream. \texttt{Goal} is the goal where the problem occurred, \texttt{Operation} is the operation to be performed (here, \texttt{output}), \texttt{ObjectType} is the type of the target object (\texttt{stream}), \texttt{Culprit} is the name of the output stream, and \texttt{Message} provides extra info about the occurred error.

\item \bti{nl/0}\\
    \noindent\bt{nl} writes a new line control character on current output stream. The predicate fails if the operation is not possible.

    \template{nl}

    \exception{error(permission\_error (Operation, ObjectType,\\
    Culprit), permission\_error(Goal, Operation, ObjectType,\\
    Culprit, Message))} if it was impossible to write on the output stream. \texttt{Goal} is the goal where the problem occurred, \texttt{Operation} is the operation to be performed (here, \texttt{output}), \texttt{ObjectType} is the type of the target object (\texttt{stream}), \texttt{Culprit} is the name of the output stream, and \texttt{Message} provides extra info about the occurred error.

\end{itemize}

%---------------------------------------------------------------------
\subsubsection{Helper Predicates}
%---------------------------------------------------------------------

\begin{itemize}

\item \bti{text\_from\_file/2}\\
    \noindent\bt{text\_from\_file(File, Text)} is true iff \bt{Text} is the text contained in the file whose name is \texttt{File}.

    \template{text\_from\_file(+string, -string)}

    \exception{error(instantiation\_error, instantiation\_error(\\
    Goal, ArgNo))} if \texttt{File} is a variable. \texttt{Goal} is the goal where the problem occurred, \texttt{ArgNo} indicates the argument that caused the problem (obviously, \texttt{1}).

    \exception{error(type\_error(ValidType, Culprit), type\_error(\\
    Goal, ArgNo, ValidType, Culprit))} if \texttt{File} is not an atom. \texttt{Goal} is the goal where the problem occurred, \texttt{ArgNo} indicates the argument that caused the problem (here, \texttt{1}), \texttt{ValidType} is the expected data type (\texttt{atom}), \texttt{Culprit} is the wrong term found.

    \exception{error(existence\_error(ObjectType, Culprit),\\
    existence\_error(Goal, ArgNo, ObjectType, Culprit, Message))} if \texttt{File} does not exist. \texttt{Goal} is the goal where the problem occurred, \texttt{ArgNo} indicates the argument that caused the problem (clearly, \texttt{1}), \texttt{ObjectType} is the type of the missing object (\texttt{stream}), \texttt{Culprit} is the wrong term found and \texttt{Message} provides an error message (here, most likely \texttt{file\_not\_found}).

\item \bti{agent\_file/1}\\
    \noindent\bt{agent\_file(FileName)} is true iff \texttt{FileName} is an accessible file containing a Prolog knowledge base, and as a side effect it spawns a \tuprolog{} agent provided with that knowledge base.

    \template{agent\_file(+string)}

    \exception{} the predicate maps onto the above \texttt{text\_from\_file(File, Text)} with \texttt{File=FileName}, so the same exceptions are raised.

\item \bti{solve\_file/2}\\
    \noindent\bt{solve\_file(FileName, Goal)} is true iff \texttt{FileName} is an accessible file containing a Prolog knowledge base, and as a side effect it solves the query \texttt{Goal} according to that knowledge base.

    \template{solve\_file(+string, +goal)}

    \exception{} the predicate maps onto the above \texttt{text\_from\_file(File, Text)} with \texttt{File=FileName}, so the same exceptions are raised.

    Moreover, it also raises the following specific exceptions:

    \exception{error(instantiation\_error, instantiation\_error(\\
    Goal, ArgNo))} if \texttt{G} is a variable. \texttt{Goal} is the goal where the problem occurred, \texttt{ArgNo} indicates the argument that caused the problem (in this case, \texttt{2}).

    \exception{error(type\_error(ValidType, Culprit), type\_error(\\
    Goal, ArgNo, ValidType, Culprit))} if \texttt{G} is not a callable goal. \texttt{Goal} is the goal where the problem occurred, \texttt{ArgNo} indicates the argument that caused the problem (in this case, \texttt{2}), \texttt{ValidType} is the expected data type (\texttt{callable}), \texttt{Culprit} is the wrong term found.

\item \bti{consult/1}\\
    \noindent\bt{consult(FileName)} is true iff \texttt{FileName} is an accessible file containing a Prolog knowledge base, and as a side effect it consult that knowledge base, by adding it to current knowledge base.

    \template{consult(+string)}

    \exception{} the predicate maps onto the above \texttt{text\_from\_file(File, Text)} with \texttt{File=FileName}, so the same exceptions are raised.

    Moreover, it also raises the following specific exceptions:

    \exception{error(syntax\_error(Message), syntax\_error(Goal,\\
    Line, Position, Message))} the theory in \texttt{FileName} is not valid. \texttt{Goal} is the goal where the problem occurred, \texttt{Message} contains a description of the occurred error, \texttt{Line} and \texttt{Position} provide the line and position of the error in the theory text.

\end{itemize}

%---------------------------------------------------------------------
\subsubsection{Random Generation of Numbers}
%---------------------------------------------------------------------

\noindent The random generation of number can be regarded as a form of
I/O.

\begin{itemize}
\item \bti{rand\_float/1}\\
    \noindent\bt{rand\_float(RandomFloat)} is true iff \texttt{RandomFloat} is a float random number generated by the engine between 0 and 1.
    
    \template{rand\_float(?float)}

\item \bti{rand\_int/2}\\
    \noindent\bt{rand\_int(Seed, RandomInteger)} is true iff \texttt{RandomInteger} is an integer random number generated by the engine between 0 and \texttt{Seed}.
    
    \template{rand\_int(?integer, @integer)}
\end{itemize}


%---------------------------------------------------------------------
\section{ISOIOLibrary}
\label{sec:isoio-library}
%---------------------------------------------------------------------

The ISO specification requires a lot of I/O predicates---many more than \tuprolog{} IOLibrary supports.
%
Table \ref{fig:isoiolibrary-table} summarizes the differences between \tuprolog{} IOLibrary and the ISO specifications.

\begin{figure}
  \centering
  \includegraphics[width=7cm]{images/isoiolibrary-table}
  \caption{Comparison between the I/O predicates provided by IOLibrary and the ISO standard specification. Bold style indicates missing predicates, plain style indicates existing functionalities to be refactored, improved, or be provided with a different signature. }\label{fig:isoiolibrary-table}
\end{figure}

The main reason for such a large number of differences lays is that the ISO Prolog standard defines very general concepts for I/O handling, aimed at supporting a wide variety of I/O modes and devices. More precisely:
\begin{itemize}
  \item \textbf{Sources} represent the resources from which data are read;
  \item \textbf{Sinks} represent the resources to which data are written.
\end{itemize}
%
Sources and sinks can be file, standard input/output stream, or any other resource supported by the underlying system: the only assumption is that each resource is associated to a sequence of bytes of characters.

\textit{Stream terms} provide a logical view of sources and sinks, and are used to identify a stream in I/O predicates. A stream term is a term respecting the following constraints:
\begin{itemize}
  \item it is a ground term;
  \item it is not an atom (this requirement means to distinguish stream terms from stream aliases--see below for details);
  \item it is not used to identify other streams at the same time.
\end{itemize}
%
The ISO standard does not specify whether the stream terms must result from an explicit source/sink opening by the \texttt{open/4} predicate, nor whether different sources/sinks must be represented by different stream terms at different, subsequent times: these issues are left to the specific implementation.

Moreover, each stream can be associated to a \textit{stream alias}, that is an atom used to refer to the stream. The association between a stream and its alias is created when the stream is opened, and automatically canceled when the stream is closed.
The same stream can be associated to multiple aliases simultaneously.

Two pre-defined streams exist that are always automatically open, 
the standard input (alias \texttt{user\_input}) and the standard output
(alias \texttt{user\_output}): such streams must never be closed.

The ISO standard also introduces the concepts of \textit{current input stream} and \textit{current output stream}: initially, they default to the standard input and standard output above, but can be reassigned at any time via the \texttt{set\_input/1} e and \texttt{set\_output/1} predicates.
However, when such an input/output stream is closed, the current input/output stream must be re-set to their default values (i.e., the standard input/output, respectively).

One further concept is the \textit{stream position}, which defines the point where the next input/output will takes place. The position can be changed via \texttt{set\_stream\_position/2}.
The stream position is always supported, even by predicates whose operations do not change the position itself.
Syntactically, the stream position is an implementation-dependent ground term.



%---------------------------------------------------------------------
\section{DCGLibrary}
\label{sec:dgc-library}
%---------------------------------------------------------------------

\noindent \emph{Library Dependencies}: BasicLibrary.

This library provides support for Definite Clause Grammars (DCGs) \cite{bra00}, an extension of context free grammars that have proven useful for describing natural and formal languages, and that may be conveniently expressed and executed in Prolog.
%
Note that this library is not loaded by default when a \tuprolog{} engine is created: it must be explicitly loaded by the user, or via a \texttt{load\_library} directive inside any theory using DCGs.

A DCG rule has the general form \verb|Head --> Body|: to distinguish terminal from nonterminal symbols, a phrase (that is, a sequence of terminal symbols) must be written as a Prolog list, with the empty sequence written as the empty list \verb|[]|.
%
The body can contain also executable blocks in parentheses, which are interpreted as normal Prolog rules.

Here is a simple example (see also Figure \ref{fig:dcg-example}):
%
\begin{verbatim}
    sentence --> noun_phrase, verb_phrase.
    verb_phrase --> verb, noun_phrase.
    noun_phrase --> [charles].
    noun_phrase --> [linda].
    verb --> [loves].
\end{verbatim}
%
To verify whether a phrase is correct according to the given grammar, the \texttt{phrase/2} or \texttt{phrase/3} predicates are used---the latter form providing an extra argument for the `remainder' of the input string not recognised as being part of the phrase.
%
Some examples follow:\\

\verb|?- phrase(sentence, [charles, loves, linda])|

\texttt{\textit{yes}}\\

\verb|?- phrase(sentence, [Who, loves, linda])|

\texttt{\textit{Who/charles}}

\texttt{\textit{Who/linda}}\\

\verb|?- phrase(sentence, [charles, loves, linda, but, hates, laura], R)|

\texttt{\textit{R/[but, hates, laura]}}\\

\begin{figure}
\centering
\includegraphics[width=7cm]{images/dcg-example}
\caption{The DCG Library example in the \tuprolog{} GUI (note the explicit library loading directive).}
\label{fig:dcg-example}
\end{figure}

%---------------------------------------------------------------------
\subsection{Predicates}
%---------------------------------------------------------------------

\noindent The classic built-in predicates provided for parsing DCG
sentences are:

\begin{itemize}
%
\item \bti{phrase/2}\\
    \noindent\bt{phrase(Category, List)} is true iff the list \bt{List} can be parsed as a phrase (i.e. sequence of terminals) of type \bt{Category}.
    \bt{Category} can be any term which would be accepted as a nonterminal of the grammar (or in general, it can be any grammar rule body), and must be instantiated to a non-variable term at the time of the call.
    This predicate is the usual way to commence execution of grammar rules.
    If \bt{List} is bound to a list of terminals by the time of the call, the goal corresponds to parsing \bt{List} as a phrase of type \bt{Category}; otherwise if \bt{List} is unbound, then the grammar is being used for generation.

    \template{phrase(+term, ?list)}

    \exception{error(instantiation\_error, instantiation\_error(\\
    Goal, ArgNo))} if \texttt{Category} is a variable. \texttt{Goal} is the goal where the problem occurred, \texttt{ArgNo} indicates the argument that caused the problem (obviously, \texttt{1}).

\item \bti{phrase/3}\\
    \noindent\bt{phrase(Category, List, Rest)} is true iff the segment between the start of list \bt{List} and the start of list \bt{Rest} can be parsed as a phrase (i.e. sequence of terminals) of type \bt{Category}.
    In other words, if the search for phrase Phrase is started at the beginning of list \bt{List}, then \bt{Rest} is what remains unparsed after \bt{Category} has been found.
    Again, \bt{Category} can be any term which would be accepted as a nonterminal of the grammar (or in general, any grammar rule body), and must be instantiated to a non variable term at the time of the call.

    \template{phrase(+term, ?list, ?rest)}

    \exception{error(instantiation\_error, instantiation\_error(\\
    Goal, ArgNo))} if \texttt{Category} is a variable. \texttt{Goal} is the goal where the problem occurred, \texttt{ArgNo} indicates the argument that caused the problem (obviously, \texttt{1}).

\end{itemize}

%---------------------------------------------------------------------
\subsection{Operators}
%---------------------------------------------------------------------

The full list of DCGLibrary operators, with their priority and associativity, is reported in Table \ref{tab:dcglibrary-operators}.

\begin{table}[h]
    \begin{center}{\small\tt
    \begin{tabular}{p{2cm}|p{3cm}|p{3cm}}\hline\hline
    Operator & Associativity & Priority \\ \hline
    --> & xfx & 1200\\
    \hline\hline
    \end{tabular}
    }\end{center}
    \caption{DCGLibrary operators.}\label{tab:dcglibrary-operators}
\end{table}

